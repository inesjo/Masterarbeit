

\section{Installation}
\label{sec:installation}


\subsection{Installation instructions}
\label{sec:installationInstructions}

\wegcLaTeX{} is provided in the repository hosted at
\nolinkurl{svn+ssh://wegc203117.uni-graz.at/var/lib/svn/wegc_latex/}.

The prerequisite for using \wegcLaTeX{} is a properly installed full \TeXLive{} 2012 distribution on a Linux workstation,
\IE{} the following packages:
\begin{description}
   % \item for \entity{SuSE 12.2}: \\
   %    \path{texlive}, \path{texlive-latex}, \path{texlive-bin-latex}, \\
   %    \path{texlive-tools}, \path{texlive-bin-tools}, \path{texlive-doc}, and \path{texlive-fonts-extra}
   \item for \entity{Debian 7}: \\
      \path{texlive}, and \path{texlive-full}
\end{description}

It should be noted at this place that \wegcLaTeX{} has not been tested with
\MiKTeX{} yet. Experience shows, however, that, if any, only minor
compatibility issues are to be expected.

For installing \wegcLaTeX{} it is only necessary to copy the \wegcLaTeX{}
\texmf{} tree to the \path{\plh{InstallDir}/texmf} directory, and to set up
to three environment variables. \path{\plh{InstallDir}} denotes the
directory in which \path{./texmf/} itself resides, \EG{} if the user's home
directory is used, to \path{/home/<userId>/texmf/}.

Assuming a workstation using the \entity{bash} command language
interpreter, as a second and final step, setting the three environment
variables can be done by adding the following three lines to the
\path{.bashrc} login script:
%
\begin{CommandLineListing}[style=DefaultFileListing, print=true, gobble=3]
   export TEXMFHOME=<InstallDir>/texmf
   export TEXINPUTS=.:./\{data,figs,tex\}//:../common//:
   export BIBINPUTS=.:../common//:
\end{CommandLineListing}
%
\begin{itemize}
\item The \cmdline{TEXMFHOME} variable defines the \wegcLaTeX{}
  installation directory. In this example, it is set to
  \cmdline{TEXMFHOME=<InstallDir>/texmf}. This is only necessary if
  \wegcLaTeX{} is installed to a directory other than the user's home
  (\path{/home/<userId>/}) or the system's pre-defined path
  (\path{/usr/local/share/}). The \LaTeX{} interpreter searches these two
  paths automatically and will use \wegcLaTeX{}, if found there, also
  without setting the \cmdline{TEXMFHOME} variable.
\item The \cmdline{TEXINPUTS} variable defines the locations where
  \wegcLaTeX{} searches for required packages, data, figures, or \LaTeX{}
  source code files. This is convenient, since any file in the
  \path{./data/}, \path{./figs/}, \path{./tex/} or \path{./common/}
  subdirectories can then be referred to by its basename instead of its
  (absolute or relative) pathname.
\item The \cmdline{BIBINPUTS} variable defines the locations where
  \wegcLaTeX{} searches for \path{*.bib} files.
\end{itemize}
%
\Attention{%
  You should not forget to replace the example directory names with the
  real directory names of your installation. Moreover, you must
  \enquote{source} your \path{~/.bashrc} (\IE{} by entering \cmdline{source
    ~/.bashrc} on the command line) in order for the modified shell
  environment to come into effect. If you already maintain a personal
  \texmf{} tree in a non‐standard location, say \path{~/pubs/texmf}, ensure
  that the above two commands are interpreted \emph{after} the commands
  setting up the shell environment for your personal \texmf{}
  tree. Otherwise, there is some probability that \wegcLaTeX{} will not get
  what it actually expects \ldots
}


\subsection{\wegcLaTeX{} file structure}
\label{subsec:structure}

All files belonging to the \wegcLaTeX{} framework are stored below the
\path{\plh{InstallDir}/texmf/} directory. \path{\plh{InstallDir}} denotes
the directory in which the \path{./texmf/} itself resides.
%
The file structure follows the concept of a so‐called \texmf{} tree. At first
glance, you might deem this structure unnecessarily complicated. As a
matter of fact, it is, however, most adequate, given that the file search
algorithms included in \TeX⁄\LaTeX{} distributions such as \TeXLive{} and
\MiKTeX{} are definitely designed and optimized for \texmf{} trees.

The subsequent list describes the subdirectories of the \wegcLaTeX{} file structure:
\begin{description}
   \item \path{./texmf/bibtex/} \\
      This directory hosts \entity{biblatex}-related files (\EG, style files).
    \item \path{./texmf/doc/} \\
      This directory contains the \wegcLaTeX{} templates for
      \singledoc and \multidoc documents in the \\
      \path{./texmf/doc/latex/wegc-latex/examples/singledoc/}, \\
      \path{./texmf/doc/latex/wegc-latex/examples/multidoc-article/}, \\
      \path{./texmf/doc/latex/wegc-latex/examples/multidoc-report/}, and \\
      \path{./texmf/doc/latex/wegc-latex/examples/multidoc-book/} \\
      directories, as well as the document command and layout template files in the \\
      \path{./texmf/doc/latex/wegc-latex/examples/common/} directory.
   \item \path{./texmf/doc/latex/wegc-latex/examples/singledoc/} \\
      This directory contains the \singledoc article, report and book document templates, \IE,
      \path{master-wlarticle.tex}, \path{master-wlreport.tex}, and \path{master-wlbook.tex}.
   \item \path{./texmf/doc/latex/wegc-latex/examples/multidoc-article/}, \\
         \path{./texmf/doc/latex/wegc-latex/examples/multidoc-report/}, and \\
          \path{./texmf/doc/latex/wegc-latex/examples/multidoc-book/} \\
      These directories contain \multidoc article, report and book document templates, \IE, \\
      \path{./multidoc-article/doc-article.tex}, \\
      \path{./multidoc-book/doc-book.tex}, and \\
      \path{./multidoc-report/doc-report.tex}. \\
      Each of these three subdirectories also contain a
      \path{./data/},
      \path{./figs/} and
      \path{./tex/} subdirectory for storing ASCII, image and \LaTeX{} files used.
   \item \path{./texmf/doc/latex/wegc-latex/examples/common/} \\
      This directory contains the \singledoc and \multidoc document command and layout template files \\
      \path{./common/commands.sty}, \\      
      \path{./common/docstyle.sty}, and \\
      \path{./common/project.sty}. \\
      The acronyms, terms, and address definition files 
      \path{acronyms.sty},
      \path{addresses.sty}, and
      \path{terms.sty} are also to be put into this directory 
      (example only versions of these files are provided in the compressed archive file
      \path{./texmf/doc/latex/wegc-latex/WLG/acronymsAddressesTerms_ExampleDoNotUse.tar.gz}, 
      whereas current and up to date versions of these files should be obtained from the separate 
      repository hosted at
      \nolinkurl{https://wegc203117.uni-graz.at/projects/latex_dbs/browser/arsclisys}.%%
      \footnote{The common acronyms, terms, and address
        definitions are easily accessed by adding its directory path to the
        \path{TEXINPUTS} environment variable.} \\
      %%
      The subdirectory \\
      \path{./common/figs/} \\
      contains the images used with \multidoc document cover pages and document headers.
   \item \path{./texmf/dvipdfmx/} \\
      This directory hosts configuration files for the \entity{DVIPDFMx} package,
      used for translating the \entity{DVI} format to \entity{PDF}.
   \item \path{./texmf/tex/} \\
      This directory hosts, among others, the automatically included \LaTeX{} modules mentioned previously
      and the \wegcLaTeX{} kernel.
   \item \path{./texmf/tex/<TBD>/documentVersionInfo.tex} \\
      This file contains the revision definitions used by \wegcLaTeX{}
      (intended to be updated by a \path{./configure} or \path{./makeDocument} script).
      The revision definition consists of three entities, \IE{} \\
      \cmdline{documentRevision},
      \cmdline{documentMajorVersion}, and
      \cmdline{documentMinorVersion}:
      %
      \begin{CommandLineListing}[print=true, gobble=6]
         \newcommand*{\documentMajorVersion}{5}
         \newcommand*{\documentMinorVersion}{6}
         \newcommand*{\documentRevision}{3072}
      \end{CommandLineListing}
\end{description}

For easier offline reading, the manuals for the \LaTeX{} packages mentioned
in \fullautoref{subsec:scope}, \fullautoref{subsec:capabilities} and
\fullautoref{subsec:extensionsToTheLatexKernel}, together with this manual
(\path{WLG.pdf}), are provided in the directory \path{./LaTeX_PDFs/} as
\ac{pdf} files (at the same directory level as the \wegcLaTeX{} \texmf{}
tree).



\subsection{Generating this document}
\label{subsec:generatingThisDocument}

Assuming the \wegcLaTeX{} framework is already properly installed (see \autoref{sec:installationInstructions}),
for generating this document, proceed as follows:
%%
\begin{enumerate}
   \item
      create a working directory, \EG{} \path{/home/\plh{user}/wlgDocGen/}
   \item
      checkout the latest version of \wegcLaTeX{} to \path{/home/\plh{user}/wlgDocGen/}
   \item
      change to the directory containing the master file \path{WLG.tex}, \IE{} \\
      \cmdline{cd /home/\plh{user}/wlgDocGen/trunk/texmf/doc/latex/wegc-latex/WLG}
   \item
      execute the script \path{makeWLG}, \IE{} \\
      \cmdline{./makeWLG} \\
      for generating the \entity{PDF} file \path{WLG.pdf}
\end{enumerate}
