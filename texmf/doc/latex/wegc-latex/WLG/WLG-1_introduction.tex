
\section[Introduction]{Introduction}
\label{sec:introduction}



\subsection[Scope]{Scope}
\label{subsec:scope}

The intention of this document\footnote{The document in hand is applicable
  to \software{long}{\wegcLaTeX}{0}{9}{5}{200}\p.} is to give new users of
the \wegcLaTeX{} documentation framework a quick introduction on how to use
it in the most efficient way.

Included in this manual are a series of examples on how to create the document layout and how to
use the capabilities of the automatically included \LaTeX{} packages.
No special treatment for typesetting formulas, diagrams or tables is given, as literature
on these topics is readily available.
To a minor extent, it also covers advanced topics mainly relevant to people who want to use \wegcLaTeX{}
within the scope of documentation tasks.

It must be emphasized that it is far beyond the scope of this user guide to address questions about
standard \LaTeX{} concepts. Using \LaTeX{} in a reasonably correct manner is \emph{not} a trivial task.
In fact, it is easy to use it quite wrongly.
It is expected that the reader has a basic understanding on how to create simple \LaTeX{} documents.
So if you are new to \LaTeX{} and/or have never worked in a \LaTeX{} documentation task,
please refer to introductory literature on \LaTeX{}.

At the minimum, you should read %
\textquote[\ctanurl{/tex-archive/info/lshort/english/lshort.pdf}]{%
  The Not So Short Introduction to \LaTeXe%
} %
which provides a very good survey of contemporary \LaTeX{} for both beginners and advanced users.
Nevertheless, it is recommended to study the following documents in some detail:
%
\begin{itemize}
   \item \textquote[]{\LaTeXe{} for authors}
         \footnote{\ctanurl{/tex-archive/macros/latex/doc/usrguide.pdf}}

   \item \textquote[]{An essential guide to \LaTeXe{} usage}
         \footnote{\ctanurl{/tex-archive/info/l2tabu/english/l2tabuen.pdf}}

   \item \textquote[]{\KOMAScript}
         \footnote{\ctanurl{/tex-archive/macros/latex/contrib/koma-script/doc/scrguide.pdf}}

   \item \textquote[]{Math mode}
         \footnote{\url{ftp://ftp.tex.ac.uk/tex-archive/info/math/voss/mathmode/Mathmode.pdf}}

   \item \textquote[]{Using Imported Graphics in \LaTeX{} and pdf\LaTeX}
         \footnote{\ctanurl{/tex-archive/info/epslatex/english/epslatex.pdf}}
\end{itemize}
%
Finally, download and print out the %
\textquote[\url{http://www.stdout.org/~winston/latex/latexsheet.pdf}]{%
  \LaTeXe{} Cheat Sheet%
} %
since this may serve as a handy means for everyday work with \LaTeX.

\wegcLaTeX{} has been developed and is maintained by \Name{mip}. In case of
any questions about \wegcLaTeX{}, please write an email to
\EmailAddress{mip}.



\subsection{Capabilities of \wegcLaTeX{}}
\label{subsec:capabilities}

So, what is the \wegcLaTeX{} documentation framework?
Stated in the most simple way, \wegcLaTeX{} provides an environment for generating documents with a
consistent look and feel. The consistency of the the documents is ensured by providing mechanisms
for consistent usage of common items like names, addresses, email addresses, web addresses, telephone numbers,
abbreviations, acronyms, terms and last but not least, a common and consistent document style for
\singledoc and \multidoc documents.

\smallskip

\singledoc documents have a traditional \LaTeX{} styling, whereas \multidoc
documents are intended for publications that comprise two or more articles,
reports or books, which shall all share a common and consistent layout of
the document front matter, \EG{} title page, distribution list and document
revision history as well as document headings. For a few sample pages of a
\multidoc document, please see \autoref{fig:multiDocTitlepage},
\autoref{fig:multiDocReleaseInfopage},
\autoref{fig:multiDocDistributionList}, \autoref{fig:multiDocChangeRecord}
and \autoref{fig:multiDocHeadings} in
\fullautoref{subsec:usingMultiDocumentTemplates}.

The documents written with \wegcLaTeX{} can be, for example, the
documentation tree associated with a software package, the manual and
handbook set for operating and maintenance procedures of any type of machinery, 
or single documents like \ac{msc} and \ac{phd} theses 
(please see \fullautoref{subsubsec:creatingMasterOrPhdThesis} for an expample),
scientific or technical reports, or papers.

\wegcLaTeX{} provides the means to ensure the consistent layout of the
documents by providing templates for the layout of front matter pages,
title page, distribution list, document revision history, bibliography, as
well as of the lists of figures and tables.

Due to the fact that \wegcLaTeX{} also automatically imports a series of
handy \LaTeX{} packages for the creation and inclusion of external
pictures, diagrams, verbatim text and pretty-printed source code listings,
the user does not need to load any extra packages, but can simply start
writing his or her documentation, taking the provided example documents as
a starting point.

As its name implies, \wegcLaTeX{} is written in the \LaTeX{} document
markup language which is widely used by scientists and other professionals
in both the academic and commercial world.  Distributed under the terms of
the \ac{lppl}, \LaTeX{} is free software.%
\footnote{%
  This statement has sometimes been questioned (\CF{}
  \url{http://en.wikipedia.org/wiki/LPPL}).%
} %
Owing to its platform independence, it can moreover be used in a similar
way on Linux, \ac{mac_os_x}, and Windows machines.

Contemporary \TeX/\LaTeX{} distributions such as \TeXLive{} and \MiKTeX{}
ship with a considerable variety of modules whose capabilities go far
beyond the potentials originally foreseen by the venerable \TeX{}
typesetting system and the \LaTeX{} kernel built on it. In fact, assuming
you are an experienced and sufficiently persistent \LaTeX{} user, you
should nowadays be able to cope with most tasks that might arise while
preparing scientific and/or technical documents.  An incomplete---and, of
course, subjective---list of modules providing the necessary tools for
doing so includes:
%
\begin{labeling}[→]{\KOMAScript{} bundle}
   \item[\KOMAScript{} bundle]%
      modern and highly configurable replacement for the standard \LaTeX{}
      document classes, sophisticated interface for configuring the document
      layout including page style design%
      \footnote{\ctanurl{/tex-archive/macros/latex/contrib/koma-script/doc/scrguide.pdf}}
   \item[\entity{babel} package]%
      multilingual support%
      \footnote{\ctanurl{/tex-archive/macros/latex/required/babel/base/babel.pdf}}
   \item[\entity{url} package]%
      formatting \acp{url}, email addresses, filenames, etc.%
      \footnote{\ctanurl{/tex-archive/macros/latex/contrib/url/url.pdf}}
   \item[\entity{siunitx} package]%
      formatting units of physical quantities%
      \footnote{\ctanurl{/tex-archive/macros/latex/contrib/siunitx/siunitx.pdf}}
   \item[\entity{amsmath} package]%
      high‐quality typesetting of mathematical formulae%
      \footnote{\ctanurl{/tex-archive/macros/latex/required/amslatex/math/amsldoc.pdf}}
   \item[\entity{array} package]%
      replacement for the standard \LaTeX{} tables and arrays%
      \footnote{\ctanurl{/tex-archive/macros/latex/required/tools/array.pdf}}
   \item[\entity{booktabs} package]%
      assistance in producing publication‐quality tables%
      \footnote{\ctanurl{/tex-archive/macros/latex/contrib/booktabs/booktabs.pdf}}
   \item[\entity{longtable} package]%
      creating multipage tables%
      \footnote{\ctanurl{/tex-archive/macros/latex/required/tools/longtable.pdf}}
   \item[\entity{xcolor} package]%
      colour support%
      \footnote{\ctanurl{/tex-archive/macros/latex/contrib/xcolor/xcolor.pdf}}
   \item[\entity{graphicx} package]%
      embedding external graphics given in various formats%
      \footnote{\ctanurl{/tex-archive/macros/latex/required/graphics/grfguide.pdf}}
   \item[\entity{pgf} package]%
      creating graphics%
      \footnote{\ctanurl{/tex-archive/graphics/pgf/base/doc/generic/pgf/pgfmanual.pdf}}
   \item[\entity{mhchem} package]%
      chemical formulae%
      \footnote{\ctanurl{/tex-archive/macros/latex/contrib/mhchem/mhchem.pdf}}
   \item[\entity{listings} package]%
      pretty-printing of source code%
      \footnote{\ctanurl{/tex-archive/macros/latex/contrib/listings/listings.pdf}}
   \item[\entity{glossaries} package]%
      maintaining glossaries, lists of acronyms, indices, etc.\ with
      assistance of the \path{makeindex} program%
      \footnote{\ctanurl{/tex-archive/macros/latex/contrib/glossaries/glossaries-user.pdf}}
   \item[\entity{biblatex} package]%
      maintaining bibliographies based on \BibTeX{} bibliographic database
      files with assistance of the \path{bibtex8} program%
      \footnote{\label{fnote:biblatex}\ctanurl{/tex-archive/macros/latex/exptl/biblatex/doc/biblatex.pdf}}
   \item[\entity{hyperref} package]%
      support for hyperlinks, \ac{pdf} bookmarks, \ac{pdf} document
      information, etc.%
      \footnote{\ctanurl{/tex-archive/macros/latex/contrib/hyperref/doc/manual.pdf}}
   \item[\entity{pdfpages} package]%
      support for inclusion of excerpts from \ac{pdf} documents, etc.%
      \footnote{\ctanurl{/tex-archive/macros/latex/contrib/pdfpages/pdfpages.pdf}}
\end{labeling}


Since 2008, \wegcLaTeX{}, a general purpose document preparation
framework, selects, configures, patches, and extends an adequate
subset of \LaTeX{} modules available from \ac{ctan} according to the
needs of a typical user with scientific and/or technical
background. Not surprisingly, the basis of \wegcLaTeX{} is formed by
just those modules outlined in the preceding list.\footnote{Links to
  the documentation of further \LaTeX{} packages used in \wegcLaTeX{}
  are provided with the corresponding \latexcmd{\RequirePackage{}}
  statements in the \path{wltools.sty} and \path{wlsetup.sty} style
  files of the \wegcLaTeX{} framework} To sum up once more, it can be
said that \wegcLaTeX{} represents a highlevel interface to \LaTeX{}
which should enable its user to write comprehensible, consistent,
reader-friendly, flexible, and easily maintainable scientific and/or
technical documents without being forced to spend much time on looking
into more than the (already comprehensive) fundamentals of the
\LaTeX{} document markup language.
