

\section{Commands and environments}
\label{sec:commandsAndEnvironments}


\subsection{Extensions to the \LaTeX{} kernel}
\label{subsec:extensionsToTheLatexKernel}

\begin{description}
\item[\latexcmd{\hyphen}] \enforcenewline%
  Expands to a hyphen with less restrictive hyphenation properties as
  compared to the primitive \enquote{\latexcmd{-}} character. Using
  \latexcmd{\hyphen} within compound words lowers the risk of getting
  overfull horizontal boxes. \\
  \latexcmd{\hyphen} can also be accessed via the Unicode character at code
  point U+2010.

\item[\latexcmd{\nbhyphen}] \enforcenewline%
  Expands to a non\hyphen{}breaking \latexcmd{\hyphen}. \\
  \latexcmd{\nbhyphen} can also be accessed via the Unicode character at
  code point U+2011.

\item[\latexcmd{\textendash}] \enforcenewline%
  Expands to an en-dash, slightly longer than a hyphen. This command is an
  alternative form to using \enquote{\cmdline{--}} in the \LaTeX{} source
  code, which expands to the same en-dash, but with slightly different
  hyphenation properties. The en-dash is used for ranges, \EG{} ``3--7'',
  and to separate name pairs, for example, the text%
  \begin{CommandLineListing}[print=true, xleftmargin=0pt, gobble=4]%
    ``Euler\textendash{}Lagrange, Stefan\textendash{}Sussmann''
  \end{CommandLineListing}
  gives the output: \\
  ``Euler\textendash{}Lagrange, Stefan\textendash{}Sussmann''. \\
  \latexcmd{\textendash} can also be accessed via the Unicode character at
  code point U+2013. For truly compound names (named after one person),
  such as ``Lennard-Jones'', use the simple hyphen. The en-dash is also
  used in German sentences to mark ``parenthetical comments'' (see
  \latexcmd{\textemdash} for the English style to do this), for example:
  \begin{CommandLineListing}[print=true, xleftmargin=0pt, gobble=4]%
    ``Das Wichtigste \textendash{} wenn du es wirklich wissen m\"ochtest
    \textendash{} ist, Bindestriche richtig einzusetzen.''
  \end{CommandLineListing}
  gives the output: \\
  ``Das Wichtigste \textendash{} wenn du es wirklich wissen m\"ochtest
  \textendash{} ist, Bindestriche richtig einzusetzen.'' \\
  Note that in German a space character is used to separate the en-dash
  from the surrounding words.

\item[\latexcmd{\textemdash}] \enforcenewline%
  Expands to an em-dash, slightly longer than a en-dash. This command is an
  alternative form to using \enquote{\cmdline{---}} in the \LaTeX{} source
  code, which expands to the same em-dash, but with slightly different
  hyphenation properties. Use this type of dash for ``parenthetical
  comments''. For example, the text%
  \begin{CommandLineListing}[print=true, xleftmargin=0pt, gobble=4]%
    ``The main thing\textemdash{}if you must know\textemdash{}is to use
    hyphens properly.''
  \end{CommandLineListing}
  gives the output: \\
  ``The main thing\textemdash{}if you must know\textemdash{}is to use hyphens properly.''\\
  \latexcmd{\textemdash} can also be accessed via the Unicode character at
  code point U+2014. For details about how to use the various types of
  dashes, please refer to \EG{} \url{https://en.wikipedia.org/wiki/Dash}.

\item[\latexcmd{\textminus}] \enforcenewline%
  Use this to indicate negative numbers. For example, the text
  \begin{CommandLineListing}[print=true, xleftmargin=0pt, gobble=4]%
    ``\textminus{}128''
  \end{CommandLineListing}
  gives the following output: \\
  `` \textminus{}128''\\
  \latexcmd{\textminus} can also be accessed via the Unicode character at
  code point U+2212.

\item[\latexcmd{\textfractionsolidus}] \enforcenewline%
  Yields a slash with alternative (perhaps more attractive) typeface. The
  hyphenation properties of \latexcmd{\textfractionsolidus} are less
  restrictive than those shown by the standard \LaTeX{} \latexcmd{\slash}
  command and, thus, help avoid overfull horizontal boxes. Using the
  primitive \enquote{\latexcmd{/}} character in ordinary textual contexts
  is in general deprecated. \\
  \latexcmd{\textfractionsolidus} can also be accessed via the Unicode
  character at code point U+2044.

\item[\latexcmd{\backtextfractionsolidus}] \enforcenewline%
  Expands to a backslash with a typeface similar to
  \latexcmd{\textfractionsolidus}.

\item[\latexcmd{\hardbreak}] \enforcenewline%
  Indicates that a line break at the immediately following space is
  undesirable, but still acceptable.

\item[\latexcmd{\enforcenewline}] \enforcenewline%
  Causes a line break, even at positions where this is normally forbidden
  (\EG{} at the start of a paragraph).

\item[\latexcmd{\p\plh{punctuationMark}}] \enforcenewline%
  Enforces the correct spacing after
  \entity{\plh{punctuationMark}}. \latexcmd{\p} supersedes the standard
  \LaTeX{} \latexcmd{\@} command in that it does not destroy kerning.

\item[\latexcmd{\dokerning{\plh{group_1}}{\plh{token}}{\plh{group_2}}}]
  \enforcenewline%
  Restores the kerning between \entity{\plh{group_1}} and
  \entity{\plh{group_2}} lost because of using \entity{\plh{token}}.

\item[\latexcmd{\SetUpHyphenation[\plh{language}]{\plh{hyphenationRules}}}]
  \enforcenewline%
  Lets you specify additional \entity{\plh{hyphenationRules}} for
  \entity{\plh{language}} or, if \entity{\plh{language}} is omitted, for
  the primary language of the document. The \entity{\plh{hyphenationRules}}
  are specified according to the scheme prescribed by the standard \LaTeX{}
  \latexcmd{\hyphenation} command. Note that \latexcmd{\SetUpHyphenation}
  may only be used in the document preamble.

\item[\latexcmd{\fullautoref{\plh{referenceLabel}}}] \enforcenewline%
  This is an interim replacement for the \latexcmd{\vref{}} command from
  the \entity{varioref}
  package\footnote{\ctanurl{/tex-archive/macros/latex/required/tools/varioref.pdf}}
  (which currently produces some undesired side effects). In addition to
  the output of the recommended \latexcmd{\autoref} command, the page
  number of the reference target is printed with \latexcmd{\fullautoref}.

\item[\latexcmd{\software{\plh{short|long}}{\plh{swName}}{\plh{swMajor}}{\plh{swMinor}}{\plh{swMicro}}{swRev}}]
  \enforcenewline%
  Expands to a software version string, \EG{} \\
  \latexcmd{\software{long}{rOPS}{7}{7}{1}{2111}} $\rightarrow$
  \software{long}{rOPS}{7}{7}{1}{2111}, or \newline
  \latexcmd{\software{short}{rOPS}{7}{8}{}{}} $\rightarrow$
  \software{short}{rOPS}{7}{8}{}{} \p.

\item[\latexcmd{\placeholder{\plh{dummyArgument}}}] \enforcenewline%
  Expands to a string indicating a dummy argument, \EG{} \\
  \latexcmd{\placeholder{telephoneNumber}} $\rightarrow$
  \placeholder{telephoneNumber} \p.

\item[\latexcmd{\plh{\plh{dummyArgument}}}] \enforcenewline%
  A shortcut to the previous command \latexcmd{\placeholder{}} is available as \latexcmd{\plh{}}, \EG{} \\
  \latexcmd{\plh{telephoneNumber}} $\rightarrow$ \plh{telephoneNumber} \p.

\item[\latexcmd{\cmdline{\plh{consoleCommandlineString}}}] \enforcenewline%
  Expands to a verbatim, fixed font printed \plh{consoleCommandlineString} as it is commonly encountered at a console commandline, \EG{} \\
  \latexcmd{\cmdline{tar -xzf data_file.tar.gz}} $\rightarrow$ \cmdline{tar
    -xzf data_file.tar.gz} \p.

\item[\latexcmd{\path{\plh{fileOrDirecoryPathName}}}] \enforcenewline%
  Expands to a verbatim, fixed font printed \path{\plh{fileOrDirecoryPathName}}, similar \\
  to \latexcmd{\cmdline}.

\item[\latexcmd{\entity{\plh{textString}}}] \enforcenewline%
  \latexcmd{\entity} is intended to be used for marking a \plh{textString}
  that represents some form of an entity.  For example, the text
  \latexcmd{\entity{Orion}} will be printed as \entity{Orion}\p.

\item[\latexcmd{\supplement{\plh{textStrings}}}] \enforcenewline%
  Expands to the \plh{textStrings} enclosed in square brackets.  For
  example, the text%
  \begin{CommandLineListing}[print=true, xleftmargin=0pt, gobble=4]%
    ``The data \supplement{and the run control files} reside in the
    \path{/data/RAOB/} directory.''
  \end{CommandLineListing}
  gives the output: \\
  ``The data \supplement{and the run control files} reside in the
  \path{/data/RAOB/} directory.''

   % \item[\latexcmd{\insi{\plh{textStrings}}}] \enforcenewline%
   %    Expands to the \plh{textStrings} enclosed in round brackets.
   %    For example \todo{figure out what the \insi{} is intended for}, the text%
   %    \begin{CommandLineListing}[print=true, xleftmargin=0pt, gobble=4]%
   %       ``I do not know what for \insi{and why} the \latexcmd{\insi} is there.''
   %    \end{CommandLineListing}
   %    gives the output: \\
   %       ``I do not know what for \insi{and why} the \latexcmd{\insi} is there.''
   %% TODO: insi is not working as described here. And I do not know how it works...


\item[\latexcmd{\Attention{\plh{blockOfText}}}] \enforcenewline%
  Expands to a boldface string reading \textbf{Attention:} followed by \plh{blockOfText}. \\
  For example, the text%
  \begin{CommandLineListing}[print=true, xleftmargin=0pt, gobble=4]%
    \Attention{%
      Please note that keeping food near any terminal device
      might cause unpredictable results during program execution!
    }
  \end{CommandLineListing}
  gives the output: \\
  \Attention{%
    Please note that keeping food near any terminal device
    might cause unpredictable results during program execution!
  }

\item[\latexcmd{\highlight{\plh{blockOfText}}}] \enforcenewline%
  Expands to a yellow highlighted \plh{blockOfText}, independent of the document version (\entity{final} | \entity{draft}).
  For example, the text%
  \begin{CommandLineListing}[print=true, xleftmargin=0pt, gobble=4]%
    \highlight{The output values are still unexpected!}
  \end{CommandLineListing}
  gives the output: \\
  \highlight{The output values are still unexpected!}

  \Attention{%
    The \latexcmd{\highlight} command does not automatically produce any linebreaks
    and it is not sensitive to \latexcmd{\enforcenewline} and \latexcmd{\newline} commands.
    As a matter of fact, it can not be used to highlight more than one line of text.
  }

\item[\latexcmd{\todo{\plh{blockOfText}}}] \enforcenewline%
  With a document version \entity{draft}, it expands to a sequentially
  numbered red text \emph{TODO} at the paper margin near the location
  where the \latexcmd{\todo} appears, and a list of \entity{TODOs},
  sequentially numbered, with a page indicator and the associated
  \plh{blockOfText} at the end of the document. With a document version \entity{final}, the \latexcmd{\todo} is completely ignored. \\
  An example \latexcmd{\todo} command could read:%
  \begin{CommandLineListing}[print=true, xleftmargin=0pt, gobble=4]%
    \todo{Table values need to be updated for the latest measurements!}
  \end{CommandLineListing}

\item[\latexcmd{\No{}}, \latexcmd{\CF{}}, \latexcmd{\EG{}}, \latexcmd{\ETC{}}, and \latexcmd{\IE{}}] \enforcenewline%
  \latexcmd{\No{}}, \latexcmd{\CF{}}, \latexcmd{\EG{}}, \latexcmd{\ETC{}}, and \latexcmd{\IE{}}
  are expanded to conveniently formatted strings, \EG{} \newline
  the \latexcmd{\No{}} 4711 is magic           ~ $\rightarrow$ ~  the \No{} 4711 is magic \newline
  is true, \latexcmd{\CF{}} the letter of      ~ $\rightarrow$ ~  is true, \CF{} the letter of \newline
  is false, \latexcmd{\EG{}} in the case of    ~ $\rightarrow$ ~  is false, \EG{} in the case of \newline
  mist, smoke, \latexcmd{\ETC{}} in air.       ~ $\rightarrow$ ~  mist, smoke, \ETC{} in air. \newline
  is always the case, \latexcmd{\IE{}} true.   ~ $\rightarrow$ ~  is always the case, \IE{} true.

\item[\latexcmd{\pdfTeX{}}, \latexcmd{\pdfLaTeX{}}, \latexcmd{\TeXLive{}}, \latexcmd{\MiKTeX{}}, \latexcmd{\AUCTeX{}}, and \latexcmd{\texmf{}}] \enforcenewline%
  \latexcmd{\pdfTeX{}}, \latexcmd{\pdfLaTeX{}}, \latexcmd{\TeXLive{}}, \latexcmd{\MiKTeX{}}, \latexcmd{\AUCTeX{}}, and \latexcmd{\texmf{}}
  are expanded to conveniently formatted strings, \EG{} \newline
  \latexcmd{\pdfTeX{}}, \latexcmd{\pdfLaTeX{}}, \latexcmd{\TeXLive{}}, \latexcmd{\MiKTeX{}}, \latexcmd{\AUCTeX{}}, and \latexcmd{\texmf{}}
  ~ $\rightarrow$ ~
  \pdfTeX{}, \pdfLaTeX{}, \TeXLive{}, \MiKTeX{}, \AUCTeX{}, and \texmf{}

\item[\latexcmd{\wegcLaTeX{}}] \enforcenewline%
  \latexcmd{\wegcLaTeX{}} is expanded to conveniently formatted string, \EG{} \newline
  the \latexcmd{\wegcLaTeX{}} framework  ~ $\rightarrow$ ~ the \wegcLaTeX{} framework
\end{description}


\subsection[Address book database]{Using the address book database}
\label{subsec:usingTheAddressBookDatabase}

As already mentioned in \autoref{subsec:capabilities}, \wegcLaTeX{}
provides a handy mechanism for consistent usage of names, addresses, email
addresses, web addresses, telephone numbers, and fax numbers via an
extension to the \KOMAScript{} bundle.

The address book entries can be used for referring to real persons as well
as to an organization or an institute.  In the latter case, the field which
stores the \enquote{last name} should be (by convention) left empty.
Address book entries are to be stored in a file named \path{addresses.sty}
in a directory which is included in the search path defined by the
\cmdline{TEXINPUTS} environment variable.

For the addresses to be expanded in a document, the command
\latexcmd{\RequirePackage{addresses}} has to be added directly behind the
\latexcmd{\documentclass} directive in the \LaTeX{} master file.

The address book entry for a fictive person ``\Name{kmf}'' could be
configured by adding the following lines to the \path{addresses.sty} file:
%
\begin{CommandLineListing}[style=DefaultFileListing, print=true, xleftmargin=0pt, gobble=3, %
                           caption={Example \latexcmd{\newaddressbookentry} address book entry in \latexcmd{addresses.sty}}, %
                           label=lst:addressBookExample]
   \newaddressbookentry{kmf}{%
     Karo%
   }{%
     K.%
   }{%
     Musterfrau%
   }{%
     Berggasse 37, A-1234 Lanenberg%
   }{%
     \tel{43}{123}{380}{8400}{}%
   }{%
     \fax{43}{123}{380}{8400}{36}%
   }{%
     \email{karo.musterfrau@gmx.at}%
   }{%
     \url{http://www.musterfrau.at}%
   }
\end{CommandLineListing}

The eight fields for the address book entry of a person, identified by its
shortcut \entity{personId}, \IE{}
\latexcmd{\newaddressbookentry{\plh{personId}}{\plh{field_1}}{\plh{field_2}} ... {\plh{field_8}}}, \\
are defined as described in \autoref{table:addressBookDefinition} and may
be used or extracted as indicated in
\autoref{table:addressBookExample}. The full name of \plh{personId} is
accessed via the command \latexcmd{\Name{\plh{personId}}}, resulting in an
output of ``\Name{kmf}'', and the short name of \plh{personId} is accessed
via the command \latexcmd{\ShortName{\plh{personId}}}, resulting in an
output of ``\ShortName{kmf}''.

Please note that the \latexcmd{\newaddressbookentry} field definitions make
use of the \LaTeX{} commands \latexcmd{\tel{}}, \latexcmd{\fax{}},
\latexcmd{\email{}}, and \latexcmd{\url{}} for properly formatting the
fields.\footnote{If it is desired to suppress the hyperlinking of an email
  or web addresses, the \LaTeX{} commands \latexcmd{\nolinkemail{}} and
  \latexcmd{\nolinkurl{}} can be used instead of \latexcmd{\email{}} and
  \latexcmd{\url{}}.}


\begin{longtable}{l p{5.0cm} p{4.0cm}}
  \caption{Using the \latexcmd{\newaddressbookentry} entries in \latexcmd{addresses.sty}} %%
   \label{table:addressBookExample} \\
   \toprule
   Item              & Example setting/access                        &  Example output    \\
   \midrule
   first name        & \path{Karo}                                   &                    \\
   abbr.\ first name & \path{K.}                                     &                    \\
   last name         & \path{Musterfrau}                             &                    \\
   address           & \path{Berggasse 37, A-1234 Lanenberg}         &                    \\
   tel.\ number      & \latexcmd{\tel{43}{123}{380}{8400}{}}         &                    \\
   fax number        & \latexcmd{\fax{43}{123}{380}{8400}{36}}       &                    \\
   email address     & \latexcmd{\email{karo.musterfrau@gmx.at}}     &                    \\
   web address       & \latexcmd{\url{http://www.musterfrau.at}}     &                    \\
   \midrule
   name              & \latexcmd{\Name{kmf}}                         & \Name{kmf}         \\
   short name        & \latexcmd{\ShortName{kmf}}                    & \ShortName{kmf}    \\
   address           & \latexcmd{\Address{kmf}}                      & \Address{kmf}      \\
   telephone number  & \latexcmd{\Telephone{kmf}}                    & \Telephone{kmf}    \\
   fax number        & \latexcmd{\Fax{kmf}}                          & \Fax{kmf}          \\
   email address     & \latexcmd{\EmailAddress{kmf}}                 & \EmailAddress{kmf} \\
   web address       & \latexcmd{\WebAddress{kmf}}                   & \WebAddress{kmf}   \\
   \bottomrule
\end{longtable}


\begin{longtable}{l p{5.2cm} p{5.0cm}}
  \caption{Definition of \latexcmd{\newaddressbookentry} entries in \latexcmd{addresses.sty}} %%
   \label{table:addressBookDefinition} \\
   \toprule
   Field                  & Content                                     & Command for access                 \\
   \midrule
   \entity{\plh{field_1}} & first name of \entity{\plh{personId}}       &                                    \\
   \entity{\plh{field_2}} & abbr.\ first name of \entity{\plh{personId}}&                                    \\
   \entity{\plh{field_3}} & last name of \entity{\plh{personId}}        &                                    \\
   \entity{\plh{field_4}} & address of \entity{\plh{personId}}          & \latexcmd{\Address{personId}}      \\
   \entity{\plh{field_5}} & tel.\ number of \entity{\plh{personId}}     & \latexcmd{\Telephone{personId}}    \\
   \entity{\plh{field_6}} & fax number of \entity{\plh{personId}}       & \latexcmd{\Fax{personId}}          \\
   \entity{\plh{field_7}} & email address of \entity{\plh{personId}}    & \latexcmd{\EmailAddress{personId}} \\
   \entity{\plh{field_8}} & web address of \entity{\plh{personId}}      & \latexcmd{\WebAddress{personId}}   \\
   \bottomrule
\end{longtable}


\subsection[Acronym database]{Using the acronym database}
\label{subsec:UsingTheAcronymDatabase}


\wegcLaTeX{} includes the \entity{glossaries} package, which provides a
handy mechanism for consistent usage of names and their acronyms.

Acronym entries are to be stored in a file named \path{acronyms.sty} in a
directory which is included in the search path defined by the
\cmdline{TEXINPUTS} environment variable.

The command \latexcmd{\RequirePackage{acronyms}} has to be added to the
\LaTeX{} master file directly behind the \latexcmd{\documentclass} command
to make the acronym definitions available in the document. In addition, the
command \latexcmd{\printglossary[type=acronym]} has to be included between
the \latexcmd{\begin{document}} and \latexcmd{\end{document}} directives.
Acronyms used in the main body of the document are then added to the
acronym list.  If one or more additional acronyms should be included in the
acronym list, irrespective of having been used or not, they can be added
via \latexcmd{\glsadd{\plh{acronymId}}} commands in the main body of the
document.

The acronym for a fictive company ``\acf{tmc}'' could be defined by adding
the following lines to the \path{acronyms.sty} file:
%
\begin{CommandLineListing}[style=DefaultFileListing, print=true, xleftmargin=0pt, gobble=3, %
                           caption={Example \latexcmd{\newacronym} entry in \latexcmd{acronyms.sty}}, %
                           label=lst:addressBookExample]
   \newacronym[description={%
     The Muppet Company \textemdash{} a respected and trusted supplier of evocative learning systems}]{tmc}{TMC}{%
     The Muppet Company\p, Inc\p.%
   }
\end{CommandLineListing}


The four fields for an acronym entry,
identified by its shortcut \entity{acronymId}, \IE{} \\
\latexcmd{\newacronym[description={\plh{field_1}}]{\plh{field_2}}{\plh{field_3}}{\plh{field_4}}}, \\
are defined as described in \autoref{table:acronymDefinition}.
%
\begin{table}%[tb!]
  \caption{Definition of \latexcmd{\newacronym} entries in \latexcmd{acronyms.sty}} %%
  \label{table:acronymDefinition}
  \begin{tabular}{l p{8.5cm} l}
   \toprule
   Field                  & Content                                             & Command                   \\
   \midrule
   \entity{\plh{field_1}} & description of acronym \entity{\plh{acronymId}} (optional).
                            This text, together with the acronym itself, is
                            used in the acronym list. If the description
                            field (together with the
                            \entity{\plh{description}} keyword)
                            is left out, \entity{\plh{field_4}} is
                            used instead.                                       &                            \\
   \entity{\plh{field_2}} & acronym identifier, \IE{} \entity{\plh{acronymId}}  &                            \\
   \entity{\plh{field_3}} & acronym used for \entity{\plh{acronymId}}           & \latexcmd{\acs{acronymId}} \\
   \entity{\plh{field_4}} & explicit text of acronym \entity{\plh{acronymId}}   & \latexcmd{\acl{acronymId}} \\
   \bottomrule
 \end{tabular}
\end{table}
%
The commands for using the acronyms are listed in the following
(\autoref{table:acronymcommands}):
\begin{longtable}{m{0.25\linewidth}m{0.69\linewidth}}
  \caption{List of acronym commands}
  \label{table:acronymcommands} \\
  \toprule
  \latexcmd{\ac{acronymId}} & Prints the acronym plus the explicit text of
  acronym when used for the first time; any further use of \latexcmd{\ac{}}
  will no longer generate the acronym plus the explicit text of the
  acronym, but the acronym only\textemdash{}unless it is reset with a
  \latexcmd{\glsreset{}} or \latexcmd{\glsresetall} command. This is the
  recommended command to use, since it will automatically guarantee the
  consistent usage of acronyms within the document. \\
  \midrule
  \latexcmd{\acs{acronymId}} & Prints the acronym only (short form). \\
  \midrule
  \latexcmd{\acl{acronymId}} & Prints the explicit text of acronym (long
  form). \\
  \midrule
  \latexcmd{\acf{acronymId}} & Prints the acronym plus the explicit text
  of acronym (full form). \\
  \midrule
  \latexcmd{\acp{acronymId}}
  \latexcmd{\acsp{acronymId}}
  \latexcmd{\aclp{acronymId}}
  \latexcmd{\acfp{acronymId}} & Same as above, but printing the plural form
  of the acronym. The default is to add the letter ``s'' to obtain the
  plural form. To override this, the plural form can be given in the
  definition of the acronym. Please refer to the \entity{glossaries} user manual for
  details.\\
  \midrule
  \latexcmd{\aces{acronymId}}
  \latexcmd{\acesp{acronymId}}
  \latexcmd{\acel{acronymId}}
  \latexcmd{\acelp{acronymId}}
  \latexcmd{\acef{acronymId}}
  \latexcmd{\acefp{acronymId}} & Same as above, but without
  hyperlinking. This is needed for moving arguments, such as those used by
  \latexcmd{\chapter}, \latexcmd{\section}, or
  \latexcmd{\caption}. Use only these commands in such environments! \\
  \bottomrule
\end{longtable}

The acronym description defined in \entity{\plh{field_1}} along with the
acronym itself (\IE{} \entity{\plh{field_3}})
is used in the acronym list, generated at the position where the \LaTeX{} command \\
\latexcmd{\printglossary[type=acronym]} is placed. If the optional
\entity{\plh{description}} specifier is not used in the definition of the
acronym, the long form is used in the acronym list instead. It is good
practice to always define the acronym description, either by simply
repeating the long form (\entity{\plh{field_4}}), or by repeating the long
form together with some additional descriptive text if needed for better
understanding. The additional text could be supplied using the command
\latexcmd{\supplement{\plh{textStrings}}}, see
\autoref{subsec:extensionsToTheLatexKernel}.

To achieve a consistent use of acronyms throughout the document, it is
recommended to almost always use the basic acronym command
\latexcmd{\ac{acronymId}} in the text. This ensures that the correct form
of the acronym (short form or full form) is used. Exceptions to this rule
might apply \EG{} for figure captions or chapter headings, where an
explicit form of the acronym can be more appropriate. Please note that in
these \LaTeX{} environments the acronym commands without hyperlinking must
be used (\autoref{table:acronymcommands}). For longer documents it might be
desirable to repeat the full form of an acronym in a new chapter even
though it was already used in a previous chapter. To achieve this, use the
\latexcmd{\glsresetall} command in-between chapters.

Some examples on how the acronym extraction commands are to be applied,
together with the expected output, are given in
\autoref{table:acronymExample}.


\begin{landscape}
\begin{longtable}{p{6.5cm} p{6.5cm} l}
   \caption{Using the \latexcmd{\newacronym} entries in \latexcmd{acronyms.sty}} %%
   \label{table:acronymExample} \\
   \toprule
   Item                                                  & Example setting/access                 &  Example output         \\
   \midrule
   description of acronym                                & \verb|The Muppet Company \textemdash{} | \newline
                                                           \verb|a respected and trusted supplier | \newline
                                                           \verb|of evocative systems|            &                         \\
   acronym identifier                                    & \verb|tmc|                             &                         \\
   acronym                                               & \verb|TMC|                             &                         \\
   explicit text of acronym                              & \verb|The Muppet Company\p, Inc\p.|    &                         \\
   %%\midrule
   acronym                                               & \verb|\acs{tmpc}|                      & \acs{tmc}               \\
   %%\midrule
   acronym, not hyperlinked, to be used \EG{} in
   \verb|\chapter|, \verb|\section| environments         & \verb|\aces{tmpc}|                     & \aces{tmc}              \\
   %%\midrule
   explicit text of acronym                              & \verb|\acl{tmpc}|                      & \acl{tmc}               \\
   %%\midrule
   acronym plus explicit text of acronym\footnote{Please note that after \latexcmd{\ac} has been used here,
             any further use of \latexcmd{\ac{}} will no longer generate the
             acronym plus the explicit text of the acronym, but the acronym
             only\textemdash{}unless it is reset with a \latexcmd{\glsreset{}}
             or \latexcmd{\glsresetall} command.}%%
                                                         & \verb|\ac{tmc}|                        & \glsreset{tmc}\ac{tmc}  \\
   %%\midrule
   acronym                                               & \verb|\ac{tmc}|                        & \ac{tmc}                \\
   %%\midrule
   acronym plus explicit text of acronym                 & \verb|\glsreset{tmc}\ac{tmc}|          & \glsreset{tmc}\ac{tmc}  \\
   %%\midrule
   acronym                                               & \verb|\ac{tmc}|                        & \ac{tmc}                \\
   %%\midrule
   acronym, plural form                                  & \verb|\acp{tmc}|                       & \acp{tmc}                \\
   %%\midrule
   acronym plus explicit text of acronym                 & \verb|\acf{tmc}|                       & \acf{tmc}               \\
   %%\midrule
   acronym plus explicit text, not hyperlinked, to be used \EG{} in
   \verb|\chapter|, \verb|\section| environments         & \verb|\acef{tmc}|                      & \acef{tmc}               \\
   \bottomrule
\end{longtable}
\end{landscape}


\subsection[Glossary database]{Using the glossary database}
\label{subsec:UsingTheGlossaryDatabase}

Since \wegcLaTeX{} includes the \entity{glossaries} package, the inclusion
of a glossary is easily accomplished.  Glossary entries are to be stored in
a file named \path{terms.sty} in a directory which is included in the
search path defined by the \cmdline{TEXINPUTS} environment variable.

For the glossary listing to appear in a document, the command
\latexcmd{\RequirePackage{terms}} has to be added directly behind the
\latexcmd{\documentclass} in the \LaTeX{} master file, the command
\latexcmd{\printglossary[type=main]} has to be included between the
\latexcmd{\begin{document}} and \latexcmd{\end{document}} directives, and
the \latexcmd{\glsadd{}} directives have to be added in the main document
body.

A glossary entry for a fictive process of ``\gls{firlefanzation}'' could be
configured by adding the following lines to the \path{terms.sty} file:
%
\begin{CommandLineListing}[style=DefaultFileListing, print=true, xleftmargin=0pt, gobble=3, %
                           caption={Example \latexcmd{\newglossaryentry} entry in \latexcmd{terms.sty}}, %
                           label=lst:glossaryExample]
   \newglossaryentry{firlefanzation}{%
     name={firlefanzation},%
     description={The process of firlefanzation is the firlefancing of a fanz},%
     plural={firlefanzations}%
   }
\end{CommandLineListing}

The two arguments for a glossary entry,
identified by its \entity{glossaryId} in \entity{\plh{field_1}}, \IE{} \\
%%\latexcmd{\newglossaryentry{\plh{field_1}}{\plh{setting_1=value_1}, \plh{setting_2=value_2}, \plh{setting_3=value_3}}}, \\
\latexcmd{\newglossaryentry{\plh{field_1}}{\plh{setting_1=value_1}, ..., \plh{setting_3=value_3}}}, \\
are defined as described in \autoref{table:glossaryDefinition}.

As can be seen from the example \autoref{lst:glossaryExample}, the
definition consists of a first argument, the unique label
\entity{\plh{glossaryId}} used to identify the term, and the second
argument, which is a key=value comma separated list \entity{key=value}
pairs used to set the required information for the term.  The principle
keys are \entity{name}, \entity{description} and \entity{plural}.


\begin{longtable}{l p{7.5cm} l}
   \caption{Definition of \latexcmd{\newglossaryentry} entries in \latexcmd{terms.sty}} %%
   \label{table:glossaryDefinition} \\
   \toprule
   Field                    & Content                                                   & Keyword for setting values   \\
   \midrule
   \entity{\plh{field_1}}   & glossary identifier, \IE{} \entity{\plh{glossaryId}}      &                              \\
   \entity{\plh{setting_1}} & name associated with the \entity{\plh{glossaryId}}        & \latexcmd{name=}             \\
   \entity{\plh{setting_2}} & description associated with the \entity{\plh{glossaryId}} & \latexcmd{description=}      \\
   \entity{\plh{setting_3}} & plural of \entity{\plh{glossaryId}}'s name                & \latexcmd{plural=}           \\
   \bottomrule
\end{longtable}


Once a term is defined, it can be used in the document. The main commands
for accessing the glossary entries are:

\begin{description}
   \item \latexcmd{\gls{\plh{glossaryId}}} \\
      This prints the term associated with \plh{glossaryId},
      \EG{} ``\gls{firlefanzation}''.
   \item \latexcmd{\glspl{\plh{glossaryId}}} \\
      This prints plural of the term associated with \plh{glossaryId},
      \EG{} ``\glspl{firlefanzation}''.
   \item \latexcmd{\Gls{\plh{glossaryId}}} \\
      This prints the term associated with \plh{glossaryId}, with the first character
      converted to upper case, \EG{} ``\Gls{firlefanzation}''.
   \item \latexcmd{\Glspl{\plh{glossaryId}}} \\
      This prints plural of the term associated with \plh{glossaryId}, with the first
      character converted to upper case, \EG{} ``\Glspl{firlefanzation}''.
\end{description}
