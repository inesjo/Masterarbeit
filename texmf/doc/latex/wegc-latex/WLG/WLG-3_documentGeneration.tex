
\section{Document generation}
\label{sec:documentGeneration}



\subsection{Commands for document generation}
\label{subsec:commandsForDocumentGeneration}

Assuming that a complete document has been properly written, all the required subdocument \LaTeX{} source code files, data and figure
files are present in the appropriate subdirectories (as described in \autoref{subsec:structure} on \autopageref{subsec:structure}),
and the master \LaTeX{} file is named \path{masterDoc.tex},
then there are 4 steps necessary for generating the print-ready \entity{PDF} document:

\begin{enumerate}

   \item
   \begin{enumerate}
      \item
         \label{item:pdf_mode}
         \cmdline{pdflatex masterDoc} \\
         Generate the output document in \entity{PDF} format by calling
         \pdfTeX{} in \entity{PDF} mode.  In this case,
         \latexcmd{\documentclass} should include the \cmdline{"pdftex"}
         output driver option. This is the recommended (and default) way.
      \item
         \label{item:dvi_mode}
         \cmdline{latex masterDoc && dvipdfmx masterDoc} \\
         Alternatively, generate the output document in \entity{PDF} format by calling either \TeX{} or \pdfTeX{} in \entity{DVI} mode.
         In this case, \latexcmd{\documentclass} should include the \cmdline{"dvipdfmx"} output driver option.
         If \pdfTeX{} is not installed this command replaces the preceding command.
         If \pdfTeX{} is installed it is an alternative to the preceding command, allowing to generate the output document in a way
         closer to how it would be generated by means of the traditional \TeX{} engine.
         Note that this may imply a significantly smaller \entity{PDF} file size.
   \end{enumerate}

   \item
      \label{item:glossaries}
      \cmdline{makeglossaries masterDoc} \\
      Generate \LaTeX{} code for formatting the glossary sections. Which
      index processor is internally called by the makeglossaries program
      depends on the value of the \cmdline{"glossaries/processor"}
      \latexcmd{\documentclass} option.  It can be \cmdline{"makeindex"} or
      \cmdline{"xindy"}. Default is \cmdline{"makeindex"}.

   \item
   \begin{enumerate}
      \item
         \label{item:bibliographyBT8}
         \cmdline{bibtex8 -c <csfile> -W masterDoc} \\
         Generate \LaTeX{} code for formatting the bibliography section by
         means of the \cmdline{bibtex8} program.  In this case,
         \latexcmd{\documentclass} should set the
         \cmdline{"biblatex/backend"} option to \cmdline{"bibtex8"} (this
         is the default). \cmdline{<csfile>} denotes a \BibTeX{} character
         set and sort definition file such as \path{ascii.csf},
         \path{latin1.csf} or \path{latin9.csf}.  This should match the
         encoding of the bibliographic database. Usually you do not need to
         explicitly set the \cmdline{<csfile>}, leaving you with the
         recommended way of formatting the bibliography:\\\cmdline{bibtex8
           -W masterDoc}.
      \item
         \label{item:bibliographyBT}
         \cmdline{bibtex masterDoc} \\
         Generate \LaTeX{} code for formatting the bibliography section by
         means of the bibtex program. In this case,
         \latexcmd{\documentclass} should set the
         \cmdline{"biblatex/backend"} option to \cmdline{"bibtex"}.
   \end{enumerate}

 \item Repeat \autoref{item:pdf_mode} or \autoref{item:dvi_mode} two more
   times to get all references and hyperlinks such as
   \latexcmd{\autoref{}}, \latexcmd{\ac{}} and/or \latexcmd{\textcite{}}
   \ETC{} properly updated.

\end{enumerate}

An example script, which uses as first argument the basename of the
\LaTeX{} master file, is presented in
\autoref{lst:documentGenerationScript}. Please note that during your daily
working routine, it is not necessary to perform all these steps; in most
cases, a simple run of \autoref{item:pdf_mode} will update your
\entity{PDF} most efficiently.

\Attention{%
  The more traditional \path{tex}→\path{dvi}→\path{pdf} document generation
  approach listed under \autoref{item:dvi_mode} does not work well at the
  moment because a few \LaTeX{} modules loaded by \wegcLaTeX{}
  insufficiently support the underlying \entity{dvipdfm} driver.
  Furthermore note that the \entity{glossaries} package does not give any
  indication whether \path{makeindex} must be called at a specific stage of
  document generation to update the \enquote{\glossaryname} and
  \enquote{\acronymname} sections as outlined under
  \autoref{item:glossaries}.  Unfortunately, this circumstance makes the
  implementation of an \emph{intelligent} build system a rather challenging
  task.  Assuming the \enquote{\bibname} section is very large and⁄or
  several secondary reference lists exist, calling \path{bibtex8} is likely
  to result in an error despite specifying the high‐capacity
  \path{-W} switch suggested under
  \autoref{item:bibliographyBT8}. If this happens, resort to the
  \entity{biblatex} package documentation (see \autoref{fnote:biblatex})
  which describes in detail how to maximize the capacity of the
  \path{bibtex8} program at run time.}


\begin{CommandLineListing}[style=DefaultFileListing, print=true, basicstyle={\ttfamily\small}, %
                           basewidth=0.47em, xleftmargin=0pt, gobble=3, %
                           caption={Example script for document generation}, %
                           label=lst:documentGenerationScript]
   #! /bin/bash
   export TEXMFHOME=/home/<userId>/texmf
   export TEXINPUTS=.:./\{data,figs,tex\}//:../common//:
   export BIBINPUTS=.:../common//:

   masterDoc=${1}
   latex=pdflatex
   makeglossaries=makeglossaries
   bibtex='bibtex8 -W'
   rm='/bin/rm -f'
   intermediateFiles='*.acn *.acr *.alg *.aux *.bbl *.blg *-blx.bib *.dvi *.glo *.glg *.gls *.ist *.lof *.log *.lol *.lot *.nlg *.noa *.not *.run.xml *.toc'
   ${rm} ${intermediateFiles} &&
   ${latex} ${masterDoc} &&
   ${makeglossaries} ${masterDoc} &&
   ${bibtex} ${masterDoc} &&
   ${latex} ${masterDoc} &&
   ${makeglossaries} ${masterDoc} &&
   ${latex} ${masterDoc} &&
   ${latex} ${masterDoc} &&
   ${rm} ${intermediateFiles}
\end{CommandLineListing}



\subsection{Global Options for document generation}
\label{subsec:globalOptionsForDocumentGeneration}

In \LaTeX{}, global options are specified in the optional argument of the \latexcmd{\documentclass} command which selects the
document class to be used for a document, usually found at the very beginning of a \LaTeX{} master file.
Global options are not only processed by the document class module, but are also taken into account by subsequently loaded packages.
Thus, they provide the primary means for a \LaTeX{} user to change the overall appearance and other global properties of a document.

An example \latexcmd{\documentclass} command setup for use with \wegcLaTeX{} is given in \fullautoref{lst:documentClassExample} and
a short explanation of the \latexcmd{\documentclass} command values is given in \fullautoref{table:documentClassCommandSettings}.


\begin{landscape}
\begin{CommandLineListing}[style=DefaultFileListing, print=true, basicstyle={\ttfamily\small}, %
                           basewidth=0.47em, xleftmargin=0pt, gobble=3, %
                           caption={Example \latexcmd{\documentclass} command settings in the masterfile}, %
                           label=lst:documentClassExample]
   \documentclass[
     %% output driver ("pdftex" when calling pdfLaTeX or "dvipdfmx" when calling LaTeX):
     pdftex,
     %% final or draft document version ("final" or "draft"):
     draft,
     %% web or print document version ("web" or "print"):
     web,
     %% main document language ("english", "USenglish", "UKenglish", "ngerman" or "naustrian"):
     UKenglish,
     %% paper size (ISO 216 paper size, North American paper size, "portrait" or "landscape"):
     paper=a4,
     %% font size (in pt):
     fontsize=11pt,
     %% DIV factor ("default", "calc" or integer >= 4):
     DIV=11,
     %% binding correction (in mm):
     BCOR=0mm,
     %% way of including glossary section headings in the table of contents ("nottotoc", "totoc" or "totocnumberline"):
     glossaries=totoc,
     %% acronym style ("default", "dua", "footnote", "smallcaps", "smaller", "description", "description+dua" or
     %% "description+footnote"):
     glossaries/acronymstyle=default,
     %% index processor used along with the glossaries package ("default", "makeindex" or "xindy"):
     glossaries/processor=makeindex,
     %% bibliography/citation style ("default" or a style known to the biblatex package):
     biblatex/style=default,
     %% bibliographic database backend used along with the biblatex package ("default", "bibtex" or "bibtex8"):
     biblatex/backend=bibtex8,
     %% encoding of the bibliographic database ("default", "auto", "x-ascii", "x-iso-8859-15" or another
     %% single-byte encoding known to the inputenx package):
     biblatex/bibencoding=x-ascii
   ]{wlarticle}[2011/07/26]
\end{CommandLineListing}
\end{landscape}

\begin{longtable}{l p{6.0cm} p{3.0cm}}
   \caption{Available \latexcmd{\documentclass} command settings in the masterfile} %%
   \label{table:documentClassCommandSettings} \\
   \toprule
   Command item        &  possible values                                                      & example setting \\
   \midrule
   document class      & \entity{wlarticle} | \entity{wlbook} | \entity{wlreport}              & \path{wlreport} \\
   output driver       & \entity{pdftex} | \entity{dvipdfmx}                                   & \path{pdftex} \\
   document version    & \entity{final} | \entity{draft}                                       & \path{final} \\
   document type       & \entity{web} | \entity{print}                                         & \path{print} \\
   document language   & \entity{english} | \entity{UKenglish} | \entity{USenglish} | \newline
                         \entity{ngerman} | \entity{naustrian}                                 & \path{UKenglish} \\
   paper size          & \entity{a4} | \entity{a3}                                             & \path{paper=a4} \\
   font size           & \entity{11pt}                                                         & \path{fontsize=11pt} \\
   \entity{DIV} factor & \entity{default} | \entity{calc} | integer >= 4                       & \path{DIV=11} \\
   binding correction  & binding correction (in mm)                                            & \path{BCOR=5mm} \\
   \bottomrule
\end{longtable}


\subsection{Merging subdocuments}
\label{subsec:mergingSubdocuments}

For merging and joining partial documents or subdocuments to a single
\entity{PDF} document, the \entity{multivalent} tool
(\url{http://multivalent.sourceforge.net/}) is highly recommended.  This
Java based tool requires an up to date Sun-Java installation and can be
downloaded\footnote{The file \path{Multivalent20091027.jar} and an older
  version of this tool (\path{Multivalent20060102.jar}) are stored in the
  directory \path{./LaTeX_PDFs/Multivalent/} for ease of access only.}
%%
from the \entity{multivalent} download page at
\url{http://sourceforge.net/projects/multivalent/files/multivalent/Release20091027/Multivalent20091027.jar/}.\\
For joining two \entity{PDF} files, perform the following steps:
%
\begin{CommandLineListing}[print=true, gobble=0]
   cp Multivalent20060102.jar /usr/local/share/java/
   export CLASSPATH=/usr/local/share/java/*
   java tool.pdf.Merge file1.pdf file2.pdf
\end{CommandLineListing}

Note that for simple applications such as adding one or several cover pages
produced by an external tool in \entity{PDF} format, the \entity{pdfpages}
package included in \wegcLaTeX{} (\autoref{subsec:capabilities}) will be a
more appropriate approach. Please refer to the \entity{pdfpages} manual for
further details.
